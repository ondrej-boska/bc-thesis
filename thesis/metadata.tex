%%% Please fill in basic information on your thesis, which will be automatically
%%% inserted at the right places.

% Type of your thesis:
%	"bc" for Bachelor's
%	"mgr" for Master's
%	"phd" for PhD
%	"rig" for rigorosum
\def\ThesisType{bc}

% Language of your study programme:
%	"cs" for Czech
%	"en" for English
\def\StudyLanguage{cs}

% Thesis title in English (exactly as in the official assignment)
% (Note: \xxx is a "ToDo label" which makes the unfilled visible. Remove it.)
\def\ThesisTitle{Nature inspired algorithms for demand-responsive transport}

% Author of the thesis (you)
\def\ThesisAuthor{Ondřej Boška}

% Year when the thesis is submitted
\def\YearSubmitted{2024}

% Name of the department or institute, where the work was officially assigned
% (according to the Organizational Structure of MFF UK in English,
% see https://www.mff.cuni.cz/en/faculty/organizational-structure,
% or a full name of a department outside MFF)
\def\Department{Department of Theoretical Computer Science and Mathematical Logic}

% Is it a department (katedra), or an institute (ústav)?
\def\DeptType{Department}

% Thesis supervisor: name, surname and titles
\def\Supervisor{RNDr. Jiří Fink, Ph.D.}

% Supervisor's department (again according to Organizational structure of MFF)
\def\SupervisorsDepartment{Department of Theoretical Computer Science and Mathematical Logic}

% Study programme (does not apply to rigorosum theses)
\def\StudyProgramme{Computer Science}

% An optional dedication: you can thank whomever you wish (your supervisor,
% consultant, who provided you with tea and pizza, etc.)
\def\Dedication{%
I would like to thank my supervisor, RNDr. Jiří Fink, Ph.D., for his guidance, patience, and helpful insights.

Computational resources were provided by the e-INFRA CZ project (ID:90254), supported by the Ministry of Education, Youth and Sports of the Czech Republic.
}

% Abstract (recommended length around 80-200 words; this is not a copy of your thesis assignment!)
\def\Abstract{
This thesis explores demand-responsive transport, where vehicles pick up and drop off passengers based on individual requests. We present a model of the Dial-A-Ride Problem (DARP), which uses real road networks from OpenStreetMaps. Customers ask for rides between two locations, providing their preferred departure time. The goal is to minimize both the operating cost and the customer waiting time. We implement three different encodings of an individual for genetic algorithms and three Ant Colony Optimization frameworks. We compare the results of these algorithms on our custom generated datasets.
}

% 3 to 5 keywords (recommended) separated by \sep
% Keywords are useful for indexing and searching for the theses by topic.
\def\ThesisKeywords{%
genetic algorithm\sep ant colony optimization\sep demand responsive transport\sep metaheuristics
}

% If any of your metadata strings contains TeX macros, you need to provide
% a plain-text version for use in XMP metadata embedded in the output PDF file.
% If you are not sure, check the generated thesis.xmpdata file.
\def\ThesisAuthorXMP{\ThesisAuthor}
\def\ThesisTitleXMP{\ThesisTitle}
\def\ThesisKeywordsXMP{\ThesisKeywords}
\def\AbstractXMP{\Abstract}

% If your abstracts are long and do not fit in the infopage, you can make the
% fonts a bit smaller by this setting. (Also, you should try to compress your abstract more.)
\def\InfoPageFont{}
%\def\InfoPageFont{\small}  % uncomment to decrease font size

% If you are studing in a Czech programme, you also need to provide metadata in Czech:
% (in English programmes, this is not used anywhere)

\def\ThesisTitleCS{Přírodou inspirované algoritmy pro poptávkovou dopravu}
\def\DepartmentCS{Katedra teoretické informatiky a matematické logiky}
\def\DeptTypeCS{Katedra}
\def\SupervisorsDepartmentCS{Katedra teoretické informatiky a matematické logiky}
\def\StudyProgrammeCS{Informatika}

\def\ThesisKeywordsCS{%
evoluční algoritmy\sep ant colony optimization\sep poptávková doprava\sep metaheuristiky
}

\def\AbstractCS{%
Práce se věnuje dopravě reagující na poptávku, kde vozidla vyzvedávají a vysazují pasažéry na základě individuálních požadavků. Představujeme model problému Dial-A-Ride (DARP), který využívá reálné silniční sítě z OpenStreetMaps. Cestující žádají dopravu mezi dvěma zastávkami a specifikují svůj požadovaný čas odjezdu. Cílem je minimalizovat jak provozní náklady, tak čekací dobu zákazníků. Implementujeme 3 různé kódování jedince v evolučním algoritmu a 3 frameworky Ant Colony Optimization. Výsledky těchto algoritmů porovnáváme na našich vlastních generovaných datech.
}
