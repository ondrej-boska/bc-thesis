\chapter*{Introduction}
\addcontentsline{toc}{chapter}{Introduction}

Optimization of transportation-related problems has become a popular topic for many researchers in recent years~\cite{surveytransport}. The probably most famous \textit{Travelling Salesman Problem (TSP)} focuses on finding the shortest path to visit all given cities and return back to the origin~\cite{surveytsp}. Over time, more sophisticated models were developed, like the \textit{Vehicle Routing Problem (VRP)}. The \textit{VRP} was first proposed by Dantzig and Ramser for finding the optimal routing of multiple delivery trucks~\cite{Dantzig1959TheTD}. The goal is to minimize the travel cost, which can consist of distance or time traveled. Many other models were derived from the \textit{VRP}. In a \textit{Capacitated~VRP}, the vehicle used has a set maximum capacity, which cannot be exceeded. In a  \textit{VRP~with~Time~Windows}, each customer has a specific time interval in which they must be served. The interval can be \textit{soft}, in which case the vehicle can arrive outside of the time interval but incurs a penalty. The \textit{Periodic~VRP} can be used to optimize services like waste collection or public transport, where the vehicle routes repeat over a period of time.~\cite{VRPSurvey}.

The Dial-A-Ride Problem (\textit{DARP}) was first mentioned by Psaraftis \cite{psaraftis1980dynamic}. It is the most general form of the \textit{Vehicle Routing Problem}~\cite{darpmunk}. In this model, passengers request transport between two stops, giving either a desired time of departure from the \textit{departure stop} or a desired time of arrival from the \textit{destination stop}. The goal is to find a set of routes for a fleet of buses so that all the customer requests are satisfied while the cost of operation is minimized. All buses need to start and end at a given depot. In the most general form, there can be multiple depots, and each vehicle can start and end its route at a different one. The bus fleet can either be \textit{homogeneous}, meaning that all buses have the same capacity, or \textit{heterogeneous}, where different vehicles can have different capacities or other parameters.

Several algorithms for solving the \textit{DARP} have been proposed in recent years. Popular techniques include \textit{branch-and-cut}~\cite{branchandcutcordeau}, \textit{tabu search}~\cite{tabucordeau}, \textit{insertion based heuristics}~\cite{insertionjaw} or \textit{neighborhood search}~\cite{neighbourhoodparragh}. In this thesis, we will focus on solving the problem using \textit{nature-inspired metaheuristics}, namely \textit{genetic~algorithms} and the \textit{Ant~Colony~Optimization}.

\xxx{genetic}

\xxx{aco}

In Chapter \ref{ch:problem}, we describe in detail a specific and slightly simplified model of the \textit{DARP}. In Chapters \ref{ch:genetic} and \ref{ch:aco}, we present several implementations of nature-inspired metaheuristics to find the optimal solution for the \textit{DARP} model described earlier. Finally, in Chapter \ref{ch:experiments}, we perform a series of experiments to compare these implementations. In Appendix \ref{app:user}, we describe how to run all scripts and algorithms implemented and how to replicate the results of our experiments. In Appendix \ref{app:devel}, we briefly describe the details of the implementations.
