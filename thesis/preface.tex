\chapter*{Introduction}
\addcontentsline{toc}{chapter}{Introduction}

\xxx{Introduction of the thesis}

\newpage

\sectionwithtoc{General problem description}

The problem we will be trying to solve in this thesis is usually known as the Dial-A-Ride Problem (\textit{DARP}). For given customer requests, the goal is to find the best set of bus routes so that the operational costs are minimized while all the customers are satisfied.

\hspace{0pt}

Each customer submits their transportation request, which contains the departure and destination coordinates, the time of departure, and the group size. After all the requests are collected, bus routes are generated so that the total operational cost is minimized while the following constraints are satisfied:
\begin{itemize}
    \setlength\itemsep{0pt}
    \item Each bus starts and ends its route at the given depot.
    \item Each group must be picked up and dropped off in a given time window.
    \item Each group is handled by exactly one bus.
    \item Each group must be picked up and dropped off by a bus exactly once and as a whole.
    \item No bus carries more passengers than its capacity allows it to.
\end{itemize}
The operating cost is described in more detail in the section about the fitness function.

\sectionwithtoc{Problem specification}

\begin{itemize}
    \item While the time window constraint is usually defined as a hard constraint, we will consider it a soft constraint instead. This means that while running the optimization algorithm, solutions violating this constraint will be considered valid but penalized depending on how much the constraint is violated (with how much of a delay the group was dropped off). The goal will be then to minimize these penalties as well as the operational costs.
    \item Since time windows are only a soft constraint, each request needs to contain the (earliest possible) time of departure but not the (latest possible) time of arrival at the destination. Instead, the arrival time is calculated as the sum of the departure time and the travel time between the group's departure and destination points. These two times then define the group's resulting time window. This eliminates the possibility of having ``impossible'' requests within the input data (where the time window constraint would be violated even by a direct route).
    \item The \textit{DARP} problem has multiple variants, some of which include either having multiple bus types to choose from, multiple depots to start or end each route, or both. In our case, all the implemented algorithms work with only one depot location and with only one available bus type. \xxx{keep this item? discuss options on how to extend the algorithms for multiple bus types or depots after describing each algorithm?} 
    \item Since \textit{DARP} is a real-world problem, we will make some simplifications to the problem to simplify the problem's simulation. We won't constrain or penalize route lengths (we can assume that the buses can refill fuel between stops with no time penalty). We will also mostly ignore most of the needs of the bus driver, such as compulsory time breaks after driving for a certain amount of time or having limited working hours.
\end{itemize}

\sectionwithtoc{Fitness function}
\label{sec:fitness}

The most essential part of our nature-inspired heuristics is the fitness function, which allows us to estimate the quality of a solution and compare it with other solutions.

Since the goal is to minimize the costs, they form the foundation of the function. With $S$ being the solution, $r$ a route within the solution, $b$ the type of bus assigned to the current route, $o_{b_r}$ the operational cost of the bus per kilometer, $f_{b_r}$ the one-time fee for using the bus, $s_i$ the $i$th stop on the route and $d(s_i, s_j)$ the distance between stops $s_i$ and $s_j$ in kilometers, the total operational cost of the buses is given by the equation
\begin{equation}\label{eq:fitness_costs}
    \sum_{r \in S} ( f_{b_r} + o_{b_r} \cdot \sum_{i=1}^{|r|}d(s_{i-1},s_{i}))
\end{equation}

To account for the customer's delays when evaluating a solution. we will represent the ``satisfaction'' of the customers by the equation
\begin{equation}\label{eq:fitness_satisfaction}
     \sum_{g \in groups} p_t \cdot d_g^2
\end{equation}
where $p_t$ is the penalty constant for late arrival and $d_g$ is the group's delay. The group's delay is squared to penalize larger delays more. The penalty constant should depend on the priority of delivering all the customers as soon as possible at the expense of higher operating costs.

The delay is usually calculated as the difference between the expected arrival time (the sum of departure and travel times) and the actual arrival time.

While the problem could be perceived as a multi-objective optimization between the costs and the delays, we will consider it a simple-objective optimization by simply summing the two equations (since both equations need to be minimized), giving us the resulting fitness function.

\begin{equation}\label{eq:fitness}
    \sum_{r \in S} ( f_b + o_{km_b} \cdot \sum_{i=1}^{|r|}d(s_{i-1},s_{i})) + \sum_{g \in groups} p_t \cdot d_g^2
\end{equation}

\sectionwithtoc{Input data}

All the implemented optimization algorithms require the following instance data:
\begin{itemize}
    \item List of passenger groups. Each group comprises \textit{properties} and \textit{geometry}. Properties include the group's unique identifier $id$, time of departure $t_d$ (given as the number of seconds from the start of the simulation), and size (number of passengers in the group). Geometry includes exactly two coordinates, where the first coordinate is considered the group's departure location and the second the destination location.
    \item The depot, which consists of its coordinates, unique identifier $id$, and the list of bus types available from the depot. Every bus type must include the type's unique identifier $id$, the capacity of the bus $c$, the operating cost of the bus per kilometer $o$, and the one-time fee for using the bus $f$. There is no limit to how many buses of each type can be used in the solution. As denoted earlier, even though the data format supports having multiple bus types available in a depot, the algorithms consider only the first one.
    \item Distances between each point (given in meters).
    \item Travel times between each point (given in seconds).
\end{itemize}

The size of any group cannot exceed the capacity of the largest available bus.

The list of groups and depots is stored in a \textit{GeoJSON} file, which is described in detail in a JSON Schema \xxx{found in the attachments of this thesis \ref{lst:jsonschema}}. The distance and duration matrices are stored in separate \textit{CSV} files.

Unlike, for example, the Vehicle Routing Problem, the \textit{DARP} does not have any commonly used benchmark data (at least not to my knowledge \xxx{source \cite{darp_j_l_b}}?), so we generate it by our own data generators. While the customer requests are generated randomly using predefined rules, the distance and duration matrices are generated using the Open Source Routing Machine (\textit{OSRM}) API \cite{luxen-vetter-2011}. The data generation is described in more detail in the experimental results section. \xxx{Or sooner since we will need use the generated data when tuning hyperparameters?}

\sectionwithtoc{Solutions}

The solution is a list of routes. Each route is given as a list of group indices in the order the bus handles their pick-ups and drop-offs.

Of course, for a solution to be valid, none of the problem constraints can be violated. This means that each group can appear in only one route. Each group's index must be included in its corresponding route \textit{exactly twice} - the first occurrence marks the pick-up, and the second occurrence marks the drop-off. Also, for each group's pick-up, the bus needs to have enough free capacity to accommodate the group.

When there is only one bus type and depot, there is no need to mention them explicitly in the solution - all routes start and end at the depot and use the same bus type.