\chapter*{Introduction}
\addcontentsline{toc}{chapter}{Introduction}

Optimization of transportation-related problems has become a popular topic for many researchers~\cite{surveytransport}. The probably most famous \textit{Travelling Salesman Problem (TSP)} focuses on finding the shortest path to visit all given cities and return back to the origin~\cite{surveytsp}. Over time, more sophisticated models were developed, like the \textit{Vehicle Routing Problem (VRP)}. The \textit{VRP} was first proposed by Dantzig and Ramser for finding the optimal routing of multiple delivery trucks~\cite{Dantzig1959TheTD}. The goal is to minimize the travel cost, which can consist of distance or time traveled.

Many other models were derived from the \textit{VRP}~\cite{VRPSurvey}. For instance, in a \textit{Capacitated~VRP}, the vehicle used has a maximum capacity, which cannot be exceeded~\cite{cvrpsurv}. In a \textit{VRP~with~Time~Windows}, each customer has a specific time interval in which they must be served~\cite{vrptw}. The interval can be \textit{soft}, in which case the vehicle can arrive outside the time interval, but incurs a penalty. The \textit{Periodic~VRP} can be used to optimize services like waste collection or public transport, where the vehicle routes repeat over a period of time~\cite{pvrp2008}.

The Dial-A-Ride Problem (\textit{DARP}) was first mentioned by Psaraftis \cite{psaraftis1980dynamic}. It is the most general form of the \textit{Vehicle Routing Problem}~\cite{darpmunk}. In this model, passengers request transport between two stops, giving a desired departure time from \textit{departure location} or a desired arrival time from \textit{ destination location}. The goal is to find a set of routes for a fleet of buses so that all the customer requests are satisfied while the cost of operation is minimized. All buses need to start and end at a given depot. In a more general form, there can be multiple depots, and each vehicle can start and end its route at a different one. The bus fleet can be \textit{homogeneous}, meaning that all buses have the same capacity, or \textit{heterogeneous}, where different vehicles can have different capacities or other parameters.

Several algorithms for solving the \textit{DARP} have been proposed in recent years. Popular techniques include \textit{branch-and-cut}~\cite{branchandcutcordeau}, \textit{tabu search}~\cite{tabucordeau}, \textit{insertion based heuristics}~\cite{insertionjaw} or \textit{neighborhood search}~\cite{neighbourhoodparragh}. In this thesis, we will focus on solving the problem using \textit{nature-inspired metaheuristics}, namely \textit{genetic~algorithms} and the \textit{Ant~Colony~Optimization}.

\textit{Genetic algorithms}~\cite{Holland:1992} are an optimization technique based on the idea of natural selection. They are based on an iterative improvement of solutions called \textit{individuals}. Each individual has a \textit{fitness value} assigned to it, representing the quality of the solution. Using genetic operators like \textit{crossover} and \textit{mutation}, new individuals are created from the individuals in previous generations called \textit{parents}. The next generation is selected from these new individuals using a \textit{selection} mechanism.

\textit{Ant Colony Optimization}~\cite{Dorigo2010} is a metaheuristic inspired by the behavior of ants. Like ants laying pheromones while searching for food, the algorithm uses a \textit{pheromone matrix} to guide the search. New solutions are constructed by traversing through the \textit{search space}, where paths with higher pheromone values are more likely to be chosen. The more the ants traverse a path, the higher the pheromone value on the path is.

The goal of this thesis is to implement and compare various individual encodings for the genetic algorithm to address our \textit{DARP} model. We seek to determine how different encodings influence the genetic algorithm's performance. We then implement a variant of the Ant Colony Optimization to solve the DARP and further compare it to the genetic algorithms. We again try using different \textit{ACO} frameworks to find out how great a difference they have on optimization performance.

Our \textit{DARP} model uses real-world coordinates, with distances and durations representing actual shortest paths in road networks. The generated routes then correspond to real roads and can be visualized using \textit{geographic information systems} like \textit{QGIS}~\cite{QGIS_software}.

In Chapter \ref{ch:problem}, we describe in detail our model of the \textit{DARP}. In Chapters \ref{ch:genetic} and \ref{ch:aco}, we present several implementations of nature-inspired metaheuristics to find the optimal solution for the \textit{DARP} model described earlier. Finally, in Chapter \ref{ch:experiments}, we perform a series of experiments to compare these implementations. In Appendix \ref{app:user}, we describe how to run all scripts and algorithms implemented and how to replicate the results of our experiments. In Appendix \ref{app:devel}, we briefly describe the details of the implementations.
