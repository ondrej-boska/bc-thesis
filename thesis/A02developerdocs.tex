\chapter{Developer Documentation}\label{app:devel}

\section{Data generators}

Scripts for generating input data for the optimization algorithms are stored in the \textit{data\_generators} folder.

The \textit{utils.py} file stores general functions used by all the generators. This includes:
\begin{itemize}
    \setlength\itemsep{0pt}
    \item Sampling public transport platforms from \textit{OpenStreetMaps} using the \textit{Overpy} library and \textit{Overpass} API. The query for the API is hard-coded in the file. 
    \item Getting the distances and durations matrices using the \textit{OSRM} API. This is done by simply using the \textit{table service}\footnote{\url{https://project-osrm.org/docs/v5.24.0/api/\#table-service}} from the API.
    \item Writing the generated data to files.
\end{itemize}

The \textit{dataGeneratorUniform.py} then uses the \textit{utils} file to sample platforms from a given area, generates random attributes for the customer requests, and saves the generated data. The commute data generator in \textit{dataGeneratorCommute.py} works similarly but takes 3 areas (from, to, depot) instead. There is also a \textit{dataGeneratorSmall.py} script, which creates data with manually selected platforms and attributes. This dataset was used for testing purposes.

\section{Optimization algorithms}

All the algorithms are written in the Julia programming language. Each algorithm's source code is stored in a separate file. Other files include parsing the input data, common functions for all evolutionary algorithms (like selection or elitism), the fitness function, the runner script, configuration variables, and utility functions that did not fit elsewhere. Each of the following sections corresponds to one source file. Every function is described by a \textit{docstring} and code comments where needed, so look directly at the source files for a detailed description.

\subsection{Input data parsing}

Input data parsing is done in the \textit{input\_parser.jl} file. The data is parsed by the \texttt{load\_input} function. The function uses the \textit{CSV} and \textit{DataFrames} libraries to parse the distances and durations matrices, the \textit{GeoJSON} library to load the input GeoJSON and \textit{JSON3} and \textit{JSONSchema} libraries to validate the input GeoJSON against a JSON Schema.

The group features are stored in a custom \texttt{Group} struct, and the depot is stored in a custom \texttt{Depot} struct holding the available bus types, also stored in a custom \texttt{BusType} struct.

The function then returns the parsed data in the form of a \texttt{InputData} struct, which holds a mapping between the group's IDs and their features (in a dictionary if the IDs in the input GeoJSON were not continuous), the depot, the distances and durations matrices and the coordinates to their id mapping.

\subsection{Fitness function}

The fitness function calculation is stored in the \textit{fitness.jl} file. The fitness is evaluated using the \texttt{evaluate\_fitness} function, which takes a valid solution and an input data instance. The solution must be stored as a \texttt{Vector\{Vector\{Int\}\}}, where each vector is a route consisting of pick-ups and drop-offs of groups (the first occurrence of the group's id marks its pick-up, the second occurrence marks the drop-off). The function separately calculates the bus costs (fixed cost per bus and operating cost per kilometer) and the delay penalties. These two values are then summed and returned. The boolean \texttt{multi\_objective} parameter can be used to return these values separately in an array.

\subsection{Genetic algorithms}

Functions common for all the genetic algorithms are stored in the \textit{evolution\_base.jl} file. Implemented functions include the tournament selection, minimizing roulette wheel selection (by using the inverse of the fitness function), the method for creating an initial population from a \textit{create individual} function, methods for applying crossover and mutation on the whole population based on the given probability, and the method for applying elitism. The \texttt{run\_genetic\_algorithm} function then implements the main loop of the genetic algorithm using these functions. This function also logs the algorithm's progress. When the \texttt{verbose} parameter is set to \texttt{true}, the progress is also printed to the standard output. The \texttt{map\_fun} parameter can be used to replace the classic \texttt{map} function with e. g. the \texttt{parallel\_map} from \textit{utils.jl} to calculate the fitnesses of the population in parallel. 

Each of the 3 implemented encodings for the genetic algorithm is implemented in its separate file. In all implementations, we first define the function for creating the initial individuals. The \texttt{cross} function then implements the crossover, and the \texttt{mutation} function implements the mutation. These functions are named the same for all encodings and when running the GA, the correct ones are chosen by Julia's multiple dispatch. Every encoding also implements a \texttt{individual\_to\_solution} method to convert the individual to an instance of the solution, for which the fitness can be calculated.

\subsubsection{Individual as stops}

The individual is stored in a \texttt{Vector\{Vector\{Int\}\}}, where each of the vectors is one route.

When converting an individual to a solution, we need a data structure that returns all the groups departing from a given place. We precompute this before the GA with the \texttt{get\_departure\_place\_group\_map} function and pass it to the fitness function as a parameter.

The submutations are implemented as nested functions within the \texttt{mutation} function. When the operator is called, one of these submutations is chosen using the \texttt{sample} function from the \texttt{StatsBase} package.

\subsubsection{Individual as separate clustering and routing}

For the \textit{EVO-CR} encoding, the individual is stored in a \texttt{EvoCRIndividual} struct. The struct includes a \texttt{Vector\{Int\}} for mapping between groups and buses (routes) and a \texttt{Vector\{Int\}} for the pick up/drop off order (its length is twice the number of groups, values of the array are group's ids, each id is present exactly twice).

\subsubsection{Individual as clustering with heuristic routing}

The individual is encoded in a single \texttt{Vector\{Int\}} defining the mapping between groups and used buses.

The greedy heuristic was inspired by the implementation in \cite{Baugh1998INTRACTABILITYOT}. While the original approach worked with \textit{heuristic depth} fixed to $4$, we define it as a parameter.

\subsection{Ant colony optimization}

The \textit{ACO-HCF} implementation is stored in the \textit{aco.jl} file. The file includes a \texttt{ant\_colony\_optimization} function with the main loop, \texttt{initialize\_pheromone} and \texttt{update\_pheromone} functions for working with the pheromone matrix, the \texttt{attractiveness} function and the \texttt{construct\_solution} function for creating new solutions based on the pheromones. The generated solutions are stored as \texttt{Vector\{Vector\{Int\}\}} and are then directly passed to the fitness function.

The \texttt{attractiveness} function imports the greedy heuristic from \textit{EVO-H} and namely uses the \texttt{move\_cost} function for calculating the weighted sum between travel time and time violations.

\subsection{Configuration}

The configuration variables, stored in the \textit{config.yaml} file, are parsed using the \textit{config.jl} file. The file implements functions for loading the YAML file and storing it in a dictionary, validating it against a JSON Schema, extracting specific sections from the configuration (and putting all the variables in the top level for easier usage) or updating the configuration (which returns a new copy of the dictionary and is useful for e.g. grid searching). The configuration dictionary uses Julia's \texttt{Symbol} type as keys.

\subsection{Runner}

In the \textit{EvoDARP.jl} file, a runner function is implemented for each optimization algorithm. Each function implements a \texttt{run\_once} function, which prepares necessary data structures (if needed), stores the configuration used, sets the random seed, and runs the optimization algorithm once. The runner functions run this function in parallel (using \texttt{@threads}).

Two \textit{main} functions are implemented; one is used when running the runner from the command line, and the other when running it in interactive mode (for example, in VS Code). Having the \textit{interactive main} separate allowed for easier experimenting during development and algorithm fine-tuning (using interactive mode reduced the compile time needed during the development because only changed functions get recompiled).

\section{Results parser}

The script for creating experiment reports, plots, and visualization maps is stored in the \textit{resultsParser.py} file in the folder \textit{scripts} folder. It is divided into several regions for easier navigation. Important regions are \textit{Experiment report}, \textit{Map visualization}, and \textit{Plotting}.

For creating the reports and maps, the script needs to load the input dataset. It does so by looking for the dataset folder in experiment's \textit{config.yaml}, and from that folder, it loads the data into custom \textit{dataclasses} \texttt{group}, \texttt{bus\_type} and \texttt{depot}. It also reads the best solution from \textit{best\_solution.csv}. The data loading functions are imported from a separate \textit{utils.py} file.

When creating the experiment reports, the script goes through the solution and runs a simulation for each route while storing kilometers traveled, operating costs, delays for each group, and the maximum number of people on the bus simultaneously. After simulating all the routes, the results are put together and written in a file.

There are two ``main'' functions for plotting. One creates plots for a single experiment, and the other creates comparison plots for all experiments in the root folder. They read the \textit{.fits} files with fitness values in time and store them in \textit{pandas} \texttt{Series}. The index of this series is either current generation or time (in milliseconds), depending on what should be visualized on the x-axis. The value is then the best fitness in said generation/time. The series are then compressed by removing adjacent entries with the same value if the \texttt{compress} argument is \textit{true}. (this makes resulting plots significantly smaller in size). All the series are then put together into a \texttt{DataFrame}, from which \texttt{NaN} values are removed using \texttt{ffill()}. From this \texttt{DataFrame}, the final statistics (mean, first, and third quartile) are then calculated. Those statistics are then plotted using the \textit{Matplotlib} library and stored in a \textit{.pdf} file.

To visualize the routes, the script first uses the \textit{OSRM} API to get the routes as GeoJSON LineStrings. The API has a convenient \textit{route service}\footnote{\url{https://project-osrm.org/docs/v5.24.0/api/\#route-service}} for this. The \textit{Folium} library is used to create a simple visualization in HTML. The routes are also stored in a GeoJSON together with the stops for more complex visualizations later.