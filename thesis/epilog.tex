\chapter*{Conclusion}
\addcontentsline{toc}{chapter}{Conclusion}

In this thesis, we implemented multiple approaches how to solve our model of the \textit{Dial-A-Ride Problem}. We introduced three different encodings of an individual representing a \textit{DARP} solution for a \textit{genetic algorithm}, each with its own crossover and mutation operators. We also designed a function for creating solutions and an attractiveness function for the \textit{Ant Colony Optimization} technique. We tried using these functions in three different known \textit{ACO} frameworks and compared the results. We then compared all the presented approaches of the \textit{DARP} optimization using our own generated datasets.

Our \textit{ACO} algorithm performs the best for datasets, where dropping groups right after their pick-up resulted in the lowest cost. This was the case with less dense \textit{uniformly distributed} datasets. It is, however, a suboptimal choice for datasets, where picking up multiple groups without dropping them off immediately is essential, like in the \textit{commute} datasets. Overall, the algorithm appears to be the least universal of all the approaches presented.

The three \textit{genetic algorithms} seem to be much more balanced, giving good results for all types of datasets. All the approaches strongly depend on the hyperparameter for setting the maximum number of buses used. When set too high, the algorithms have limited ways to lower the number of buses themselves. However, setting the limit lower can result in further minimization of operating costs, of course at the expense of customer satisfaction. In our experiments, the \textit{EVO-STOPS} generally returned the worst results of all three. Using lists of stops proved to be ineffective, as the search space is much larger because buses can visit the same stop multiple times. The \textit{EVO-H} approach returned very good results for all types of datasets. It also managed to converge faster than the other two and performed better on larger datasets. The \textit{EVO-CR}, however, returned results of very similar quality, sometimes even surpassing the \textit{EVO-H}. We are pleasantly surprised by its performance since the encoding does not use complicated crossover or mutation operators. Its transformation from an individual to a solution is also the simplest, without the need for any additional heuristics.

For future work, the algorithms could be tested and used on more datasets. One scenario could include commuting from densely populated areas to a single place, for example, a school or a workplace. The effect of some hyperparameters, such as the delay penalty constant in the fitness function, could be further tested. The algorithms could also be further extended to more complex models, being able to support multiple depots or heterogeneous fleets.

The implemented genetic algorithms have a great disadvantage as they need a good estimate of the number of buses used beforehand to return good solutions. Ideally, the algorithms should be universal for every dataset, and therefore, this number should be set automatically. This could be done perhaps by using a heuristic approach to get an estimate. We could also find a reasonably small range for this parameter and run the algorithms multiple times, each time with a different bus limit.

As the \textit{EVO-CR} approach seems promising, some more complex genetic operators could be thought of and tested to help the algorithm converge faster.

For the \textit{ACO}, other and more complex variants of the \textit{attractiveness} function could be tested, to help the algorithm with datasets where the drop-off should not be immediate.